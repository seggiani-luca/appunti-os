\documentclass[a4paper,11pt]{article}
\usepackage[a4paper, margin=8em]{geometry}

% usa i pacchetti per la scrittura in italiano
\usepackage[french,italian]{babel}
\usepackage[T1]{fontenc}
\usepackage[utf8]{inputenc}
\frenchspacing 

% usa i pacchetti per la formattazione matematica
\usepackage{amsmath, amssymb, amsthm, amsfonts}

% usa altri pacchetti
\usepackage{gensymb}
\usepackage{hyperref}
\usepackage{standalone}

\usepackage{colortbl}

\usepackage{xstring}
\usepackage{karnaugh-map}

% imposta il titolo
\title{Appunti Sistemi Operativi}
\author{Luca Seggiani}
\date{2025}

% imposta lo stile
% usa helvetica
\usepackage[scaled]{helvet}
% usa palatino
\usepackage{palatino}
% usa un font monospazio guardabile
\usepackage{lmodern}

\renewcommand{\rmdefault}{ppl}
\renewcommand{\sfdefault}{phv}
\renewcommand{\ttdefault}{lmtt}

% circuiti
\usepackage{circuitikz}
\usetikzlibrary{babel}

% testo cerchiato
\newcommand*\circled[1]{\tikz[baseline=(char.base)]{
            \node[shape=circle,draw,inner sep=2pt] (char) {#1};}}

% disponi il titolo
\makeatletter
\renewcommand{\maketitle} {
	\begin{center} 
		\begin{minipage}[t]{.8\textwidth}
			\textsf{\huge\bfseries \@title} 
		\end{minipage}%
		\begin{minipage}[t]{.2\textwidth}
			\raggedleft \vspace{-1.65em}
			\textsf{\small \@author} \vfill
			\textsf{\small \@date}
		\end{minipage}
		\par
	\end{center}

	\thispagestyle{empty}
	\pagestyle{fancy}
}
\makeatother

% disponi teoremi
\usepackage{tcolorbox}
\newtcolorbox[auto counter, number within=section]{theorem}[2][]{%
	colback=blue!10, 
	colframe=blue!40!black, 
	sharp corners=northwest,
	fonttitle=\sffamily\bfseries, 
	title=Teorema~\thetcbcounter: #2, 
	#1
}

% disponi definizioni
\newtcolorbox[auto counter, number within=section]{definition}[2][]{%
	colback=red!10,
	colframe=red!40!black,
	sharp corners=northwest,
	fonttitle=\sffamily\bfseries,
	title=Definizione~\thetcbcounter: #2,
	#1
}

% disponi codice
\usepackage{listings}
\usepackage[table]{xcolor}

\definecolor{codegreen}{rgb}{0,0.6,0}
\definecolor{codegray}{rgb}{0.5,0.5,0.5}
\definecolor{codepurple}{rgb}{0.58,0,0.82}
\definecolor{backcolour}{rgb}{0.95,0.95,0.92}

\lstdefinestyle{codestyle}{
		backgroundcolor=\color{black!5}, 
		commentstyle=\color{codegreen},
		keywordstyle=\bfseries\color{magenta},
		numberstyle=\sffamily\tiny\color{black!60},
		stringstyle=\color{green!50!black},
		basicstyle=\ttfamily\footnotesize,
		breakatwhitespace=false,         
		breaklines=true,                 
		captionpos=b,                    
		keepspaces=true,                 
		numbers=left,                    
		numbersep=5pt,                  
		showspaces=false,                
		showstringspaces=false,
		showtabs=false,                  
		tabsize=2
}

\lstdefinestyle{shellstyle}{
		backgroundcolor=\color{black!5}, 
		basicstyle=\ttfamily\footnotesize\color{black}, 
		commentstyle=\color{black}, 
		keywordstyle=\color{black},
		numberstyle=\color{black!5},
		stringstyle=\color{black}, 
		showspaces=false,
		showstringspaces=false, 
		showtabs=false, 
		tabsize=2, 
		numbers=none, 
		breaklines=true
}


\lstdefinelanguage{assembler}{ 
  keywords={AAA, AAD, AAM, AAS, ADC, ADCB, ADCW, ADCL, ADD, ADDB, ADDW, ADDL, AND, ANDB, ANDW, ANDL,
        ARPL, BOUND, BSF, BSFL, BSFW, BSR, BSRL, BSRW, BSWAP, BT, BTC, BTCB, BTCW, BTCL, BTR, 
        BTRB, BTRW, BTRL, BTS, BTSB, BTSW, BTSL, CALL, CBW, CDQ, CLC, CLD, CLI, CLTS, CMC, CMP,
        CMPB, CMPW, CMPL, CMPS, CMPSB, CMPSD, CMPSW, CMPXCHG, CMPXCHGB, CMPXCHGW, CMPXCHGL,
        CMPXCHG8B, CPUID, CWDE, DAA, DAS, DEC, DECB, DECW, DECL, DIV, DIVB, DIVW, DIVL, ENTER,
        HLT, IDIV, IDIVB, IDIVW, IDIVL, IMUL, IMULB, IMULW, IMULL, IN, INB, INW, INL, INC, INCB,
        INCW, INCL, INS, INSB, INSD, INSW, INT, INT3, INTO, INVD, INVLPG, IRET, IRETD, JA, JAE,
        JB, JBE, JC, JCXZ, JE, JECXZ, JG, JGE, JL, JLE, JMP, JNA, JNAE, JNB, JNBE, JNC, JNE, JNG,
        JNGE, JNL, JNLE, JNO, JNP, JNS, JNZ, JO, JP, JPE, JPO, JS, JZ, LAHF, LAR, LCALL, LDS,
        LEA, LEAVE, LES, LFS, LGDT, LGS, LIDT, LMSW, LOCK, LODSB, LODSD, LODSW, LOOP, LOOPE,
        LOOPNE, LSL, LSS, LTR, MOV, MOVB, MOVW, MOVL, MOVSB, MOVSD, MOVSW, MOVSX, MOVSXB,
        MOVSXW, MOVSXL, MOVZX, MOVZXB, MOVZXW, MOVZXL, MUL, MULB, MULW, MULL, NEG, NEGB, NEGW,
        NEGL, NOP, NOT, NOTB, NOTW, NOTL, OR, ORB, ORW, ORL, OUT, OUTB, OUTW, OUTL, OUTSB, OUTSD,
        OUTSW, POP, POPL, POPW, POPB, POPA, POPAD, POPF, POPFD, PUSH, PUSHL, PUSHW, PUSHB, PUSHA, 
				PUSHAD, PUSHF, PUSHFD, RCL, RCLB, RCLW, MOVSL, MOVSB, MOVSW, STOSL, STOSB, STOSW, LODSB, LODSW,
				LODSL, INSB, INSW, INSL, OUTSB, OUTSL, OUTSW
        RCLL, RCR, RCRB, RCRW, RCRL, RDMSR, RDPMC, RDTSC, REP, REPE, REPNE, RET, ROL, ROLB, ROLW,
        ROLL, ROR, RORB, RORW, RORL, SAHF, SAL, SALB, SALW, SALL, SAR, SARB, SARW, SARL, SBB,
        SBBB, SBBW, SBBL, SCASB, SCASD, SCASW, SETA, SETAE, SETB, SETBE, SETC, SETE, SETG, SETGE,
        SETL, SETLE, SETNA, SETNAE, SETNB, SETNBE, SETNC, SETNE, SETNG, SETNGE, SETNL, SETNLE,
        SETNO, SETNP, SETNS, SETNZ, SETO, SETP, SETPE, SETPO, SETS, SETZ, SGDT, SHL, SHLB, SHLW,
        SHLL, SHLD, SHR, SHRB, SHRW, SHRL, SHRD, SIDT, SLDT, SMSW, STC, STD, STI, STOSB, STOSD,
        STOSW, STR, SUB, SUBB, SUBW, SUBL, TEST, TESTB, TESTW, TESTL, VERR, VERW, WAIT, WBINVD,
        XADD, XADDB, XADDW, XADDL, XCHG, XCHGB, XCHGW, XCHGL, XLAT, XLATB, XOR, XORB, XORW, XORL},
  keywordstyle=\color{blue}\bfseries,
  ndkeywordstyle=\color{darkgray}\bfseries,
  identifierstyle=\color{black},
  sensitive=false,
  comment=[l]{\#},
  morecomment=[s]{/*}{*/},
  commentstyle=\color{purple}\ttfamily,
  stringstyle=\color{red}\ttfamily,
  morestring=[b]',
  morestring=[b]"
}

\lstset{language=assembler, style=codestyle}

% disponi sezioni
\usepackage{titlesec}

\titleformat{\section}
	{\sffamily\Large\bfseries} 
	{\thesection}{1em}{} 
\titleformat{\subsection}
	{\sffamily\large\bfseries}   
	{\thesubsection}{1em}{} 
\titleformat{\subsubsection}
	{\sffamily\normalsize\bfseries} 
	{\thesubsubsection}{1em}{}

% tikz
\usepackage{tikz}

% float
\usepackage{float}

% grafici
\usepackage{pgfplots}
\pgfplotsset{width=10cm,compat=1.9}

% disponi alberi
\usepackage{forest}

\forestset{
	rectstyle/.style={
		for tree={rectangle,draw,font=\large\sffamily}
	},
	roundstyle/.style={
		for tree={circle,draw,font=\large}
	}
}

% disponi algoritmi
\usepackage{algorithm}
\usepackage{algorithmic}
\makeatletter
\renewcommand{\ALG@name}{Algoritmo}
\makeatother

% disponi numeri di pagina
\usepackage{fancyhdr}
\fancyhf{} 
\fancyfoot[L]{\sffamily{\thepage}}

\makeatletter
\fancyhead[L]{\raisebox{1ex}[0pt][0pt]{\sffamily{\@title \ \@date}}} 
\fancyhead[R]{\raisebox{1ex}[0pt][0pt]{\sffamily{\@author}}}
\makeatother

\begin{document}

\pagestyle{fancy}
\thispagestyle{empty}
\renewcommand{\thispagestyle}[1]{}

\maketitle
\documentclass[a4paper,11pt]{article}
\usepackage[a4paper, margin=8em]{geometry}

% usa i pacchetti per la scrittura in italiano
\usepackage[french,italian]{babel}
\usepackage[T1]{fontenc}
\usepackage[utf8]{inputenc}
\frenchspacing 

% usa i pacchetti per la formattazione matematica
\usepackage{amsmath, amssymb, amsthm, amsfonts}

% usa altri pacchetti
\usepackage{gensymb}
\usepackage{hyperref}
\usepackage{standalone}

\usepackage{colortbl}

\usepackage{xstring}
\usepackage{karnaugh-map}

% imposta il titolo
\title{Appunti Sistemi Operativi}
\author{Luca Seggiani}
\date{2025}

% imposta lo stile
% usa helvetica
\usepackage[scaled]{helvet}
% usa palatino
\usepackage{palatino}
% usa un font monospazio guardabile
\usepackage{lmodern}

\renewcommand{\rmdefault}{ppl}
\renewcommand{\sfdefault}{phv}
\renewcommand{\ttdefault}{lmtt}

% circuiti
\usepackage{circuitikz}
\usetikzlibrary{babel}

% testo cerchiato
\newcommand*\circled[1]{\tikz[baseline=(char.base)]{
            \node[shape=circle,draw,inner sep=2pt] (char) {#1};}}

% disponi il titolo
\makeatletter
\renewcommand{\maketitle} {
	\begin{center} 
		\begin{minipage}[t]{.8\textwidth}
			\textsf{\huge\bfseries \@title} 
		\end{minipage}%
		\begin{minipage}[t]{.2\textwidth}
			\raggedleft \vspace{-1.65em}
			\textsf{\small \@author} \vfill
			\textsf{\small \@date}
		\end{minipage}
		\par
	\end{center}

	\thispagestyle{empty}
	\pagestyle{fancy}
}
\makeatother

% disponi teoremi
\usepackage{tcolorbox}
\newtcolorbox[auto counter, number within=section]{theorem}[2][]{%
	colback=blue!10, 
	colframe=blue!40!black, 
	sharp corners=northwest,
	fonttitle=\sffamily\bfseries, 
	title=Teorema~\thetcbcounter: #2, 
	#1
}

% disponi definizioni
\newtcolorbox[auto counter, number within=section]{definition}[2][]{%
	colback=red!10,
	colframe=red!40!black,
	sharp corners=northwest,
	fonttitle=\sffamily\bfseries,
	title=Definizione~\thetcbcounter: #2,
	#1
}

% disponi codice
\usepackage{listings}
\usepackage[table]{xcolor}

\definecolor{codegreen}{rgb}{0,0.6,0}
\definecolor{codegray}{rgb}{0.5,0.5,0.5}
\definecolor{codepurple}{rgb}{0.58,0,0.82}
\definecolor{backcolour}{rgb}{0.95,0.95,0.92}

\lstdefinestyle{codestyle}{
		backgroundcolor=\color{black!5}, 
		commentstyle=\color{codegreen},
		keywordstyle=\bfseries\color{magenta},
		numberstyle=\sffamily\tiny\color{black!60},
		stringstyle=\color{green!50!black},
		basicstyle=\ttfamily\footnotesize,
		breakatwhitespace=false,         
		breaklines=true,                 
		captionpos=b,                    
		keepspaces=true,                 
		numbers=left,                    
		numbersep=5pt,                  
		showspaces=false,                
		showstringspaces=false,
		showtabs=false,                  
		tabsize=2
}

\lstdefinestyle{shellstyle}{
		backgroundcolor=\color{black!5}, 
		basicstyle=\ttfamily\footnotesize\color{black}, 
		commentstyle=\color{black}, 
		keywordstyle=\color{black},
		numberstyle=\color{black!5},
		stringstyle=\color{black}, 
		showspaces=false,
		showstringspaces=false, 
		showtabs=false, 
		tabsize=2, 
		numbers=none, 
		breaklines=true
}


\lstdefinelanguage{assembler}{ 
  keywords={AAA, AAD, AAM, AAS, ADC, ADCB, ADCW, ADCL, ADD, ADDB, ADDW, ADDL, AND, ANDB, ANDW, ANDL,
        ARPL, BOUND, BSF, BSFL, BSFW, BSR, BSRL, BSRW, BSWAP, BT, BTC, BTCB, BTCW, BTCL, BTR, 
        BTRB, BTRW, BTRL, BTS, BTSB, BTSW, BTSL, CALL, CBW, CDQ, CLC, CLD, CLI, CLTS, CMC, CMP,
        CMPB, CMPW, CMPL, CMPS, CMPSB, CMPSD, CMPSW, CMPXCHG, CMPXCHGB, CMPXCHGW, CMPXCHGL,
        CMPXCHG8B, CPUID, CWDE, DAA, DAS, DEC, DECB, DECW, DECL, DIV, DIVB, DIVW, DIVL, ENTER,
        HLT, IDIV, IDIVB, IDIVW, IDIVL, IMUL, IMULB, IMULW, IMULL, IN, INB, INW, INL, INC, INCB,
        INCW, INCL, INS, INSB, INSD, INSW, INT, INT3, INTO, INVD, INVLPG, IRET, IRETD, JA, JAE,
        JB, JBE, JC, JCXZ, JE, JECXZ, JG, JGE, JL, JLE, JMP, JNA, JNAE, JNB, JNBE, JNC, JNE, JNG,
        JNGE, JNL, JNLE, JNO, JNP, JNS, JNZ, JO, JP, JPE, JPO, JS, JZ, LAHF, LAR, LCALL, LDS,
        LEA, LEAVE, LES, LFS, LGDT, LGS, LIDT, LMSW, LOCK, LODSB, LODSD, LODSW, LOOP, LOOPE,
        LOOPNE, LSL, LSS, LTR, MOV, MOVB, MOVW, MOVL, MOVSB, MOVSD, MOVSW, MOVSX, MOVSXB,
        MOVSXW, MOVSXL, MOVZX, MOVZXB, MOVZXW, MOVZXL, MUL, MULB, MULW, MULL, NEG, NEGB, NEGW,
        NEGL, NOP, NOT, NOTB, NOTW, NOTL, OR, ORB, ORW, ORL, OUT, OUTB, OUTW, OUTL, OUTSB, OUTSD,
        OUTSW, POP, POPL, POPW, POPB, POPA, POPAD, POPF, POPFD, PUSH, PUSHL, PUSHW, PUSHB, PUSHA, 
				PUSHAD, PUSHF, PUSHFD, RCL, RCLB, RCLW, MOVSL, MOVSB, MOVSW, STOSL, STOSB, STOSW, LODSB, LODSW,
				LODSL, INSB, INSW, INSL, OUTSB, OUTSL, OUTSW
        RCLL, RCR, RCRB, RCRW, RCRL, RDMSR, RDPMC, RDTSC, REP, REPE, REPNE, RET, ROL, ROLB, ROLW,
        ROLL, ROR, RORB, RORW, RORL, SAHF, SAL, SALB, SALW, SALL, SAR, SARB, SARW, SARL, SBB,
        SBBB, SBBW, SBBL, SCASB, SCASD, SCASW, SETA, SETAE, SETB, SETBE, SETC, SETE, SETG, SETGE,
        SETL, SETLE, SETNA, SETNAE, SETNB, SETNBE, SETNC, SETNE, SETNG, SETNGE, SETNL, SETNLE,
        SETNO, SETNP, SETNS, SETNZ, SETO, SETP, SETPE, SETPO, SETS, SETZ, SGDT, SHL, SHLB, SHLW,
        SHLL, SHLD, SHR, SHRB, SHRW, SHRL, SHRD, SIDT, SLDT, SMSW, STC, STD, STI, STOSB, STOSD,
        STOSW, STR, SUB, SUBB, SUBW, SUBL, TEST, TESTB, TESTW, TESTL, VERR, VERW, WAIT, WBINVD,
        XADD, XADDB, XADDW, XADDL, XCHG, XCHGB, XCHGW, XCHGL, XLAT, XLATB, XOR, XORB, XORW, XORL},
  keywordstyle=\color{blue}\bfseries,
  ndkeywordstyle=\color{darkgray}\bfseries,
  identifierstyle=\color{black},
  sensitive=false,
  comment=[l]{\#},
  morecomment=[s]{/*}{*/},
  commentstyle=\color{purple}\ttfamily,
  stringstyle=\color{red}\ttfamily,
  morestring=[b]',
  morestring=[b]"
}

\lstset{language=assembler, style=codestyle}

% disponi sezioni
\usepackage{titlesec}

\titleformat{\section}
	{\sffamily\Large\bfseries} 
	{\thesection}{1em}{} 
\titleformat{\subsection}
	{\sffamily\large\bfseries}   
	{\thesubsection}{1em}{} 
\titleformat{\subsubsection}
	{\sffamily\normalsize\bfseries} 
	{\thesubsubsection}{1em}{}

% tikz
\usepackage{tikz}

% float
\usepackage{float}

% grafici
\usepackage{pgfplots}
\pgfplotsset{width=10cm,compat=1.9}

% disponi alberi
\usepackage{forest}

\forestset{
	rectstyle/.style={
		for tree={rectangle,draw,font=\large\sffamily}
	},
	roundstyle/.style={
		for tree={circle,draw,font=\large}
	}
}

% disponi algoritmi
\usepackage{algorithm}
\usepackage{algorithmic}
\makeatletter
\renewcommand{\ALG@name}{Algoritmo}
\makeatother

% disponi numeri di pagina
\usepackage{fancyhdr}
\fancyhf{} 
\fancyfoot[L]{\sffamily{\thepage}}

\makeatletter
\fancyhead[L]{\raisebox{1ex}[0pt][0pt]{\sffamily{\@title \ \@date}}} 
\fancyhead[R]{\raisebox{1ex}[0pt][0pt]{\sffamily{\@author}}}
\makeatother

\begin{document}
% sezione (data)
\section{Lezione del 23-09-25}

% stili pagina
\thispagestyle{empty}
\pagestyle{fancy}

% testo
\subsection{Introduzione}
Il corso di sistemi operativi riguarda l'ultima parte dello studio delle "architetture", che è partita con l'implementazione hardware in reti logiche, è continuata con lo studio del kernel in calcolatori elettronici, e termina appunto con lo studio dei sistemi operativi. Nello specifico, si considereranno sistemi operativi derivanti dalla famiglia \textbf{UNIX}.

Argomento del corso è la conoscenza di tecniche di programmazione usate nello sviluppo del sistema operativo \textbf{multiprogrammato} (più \textit{processi} o più \textit{thread}), con riferimento particolare alla programmazione \textbf{concorrente}, lo \textbf{scheduling} e la \textbf{memoria virtuale}. 

Il corso mira a dare informazioni generiche utili allo studio di qualsiasi sistema operativo (anche non direttamente derivante da UNIX), in primis rivolte alla compresione di \textit{come mai} una certa soluzione ad un problema è migliore di altre, e quali sono le tecniche che ci permettono di sviluppare soluzioni miglori.

\subsubsection{Sistemi embedded e in tempo reale}

Ci interesseremo anche ai sistemi \textbf{embedded} e sopratutto sistemi in \textbf{tempo reale}. Questi rappresentano sistemi \textit{special-purpose} (per distinguere dai sistemi a scopo generale, \textit{general-purpose}), dove dobbiamo rispettare coi nostri algoritmi di scheduling date \textbf{scadenze} temporali.

\subsubsection{Programmazione concorrente}
Con programmazione concorrente ci riferiamo alle tecniche che ci permettono di gestire più processi che si contendono le solite risorse, adottando politiche più o meno \textit{eque} per i processi, o magari privilegiandole alcune. 
Obiettivo fondamentale sarà comunque quello di evitare \textit{stalli} o \textbf{deadlock} dati da risorse occupate.

\subsubsection{Programma del corso}
Il corso è strutturato negli argomenti: 
\begin{itemize}
	\item \textbf{Concetti introduttivi} su sistemi operativi, architetture hardware e relativi cenni storici, in particolare ci interessano dettagli come la gestione della \textit{pila} e le \textit{interruzioni};
	\item \textbf{Processi} e la loro gestione, inclusi gli algoritmi di \textit{scheduling} \textit{preemptive} e \textit{non preemptive}, \textit{prioritari} e \textit{non prioritari} (FCFS, SJF, SRTF, RR). Inoltre si tratta la schedulazione nei sistemi \textbf{hard real-time} (RM, RDF);
	\item \textbf{Sincronizzazione} dei processi, quindi \textit{programmazione concorrente}, \textit{competizone} su risorse, e \textit{scambio di informazioni fra processi} (IPC);
	\item Gestione della \textbf{memoria}, quindi \textit{memoria virtuale}, \textit{segmentazione} e \textit{paginazione};
	\item Gestione dei \textbf{dispositivi} di I/O, cioè i \textit{driver};
	\item \textbf{File system} su disco, cioè i componenti software che permettono la gestione di strutture di \textit{file}, nella loro struttura sia \textit{logica} che \textit{fisica} di implementazione nei driver e nel sistema operativo;
	\item \textbf{Sicurezza}, quindi meccanismi di \textit{protezione} fra processi, controlli sugli \textit{accessi} sia in memoria che al filesystem, con riferimento al modello della \textit{matrice degli accessi}.
\end{itemize}

\subsubsection{Meccanismi e politiche}
Una prima distinzione che possiamo fare è quella fra \textbf{meccanismo} e \textbf{politica}.
\begin{itemize}
	\item Si dice \textbf{meccanismo} ciò che effettivamente implementato, in maniera sufficientemente veloce e compatta, nel kernel, per fornire il cosidetto \textit{supporto architetturale} a risorse, dispositivi, ecc...
	\item Nei sistemi operativi ci interessano invece principalmente le \textbf{politiche}, cioè decisioni (che vanno poi implementate) su come gestire \textit{a priori} date risorse, dispositivi, ecc...
\end{itemize}

\subsubsection{Sistemi operativi}
Un \textbf{sistema operativo} è in primo luogo un \textit{programma software} che ha il compito di fare da intermediario fra l'\textit{utente} e l'\textit{hardware} di un calcolatore.

Far fronte ai bisogni dell'utente significa gestire e consentire l'accesso delle risorse ai \textit{processi} di cui l'utente necessita. In questo individuiamo come obiettivi del sistema operativo:
\begin{itemize}
	\item Eseguire i \textit{programmi utente};
	\item Rendere il sistema facile da usare;
	\item Utilizzare l'hardware in maniera efficiente.
\end{itemize}

\subsubsection{Programmi}
I programmi con cui abbiamo a che fare sono per noi \textit{liste di istruzioni} (tralasciando il fatto che queste siano codificate in linguaggio macchina o in un suo linguaggio mnemonico), ordinate ma che possono presentare salti condizionali che cambiano il normale \textit{flusso di esecuzione}.

Il \textbf{comportamento} e quindi i \textbf{risultati} di un programma dipendono sì dal codice, ma anche dai \textbf{dati} in ingresso allo stesso.
In questo possiamo dire che il programma non esiste mai da solo ed è solo la parte \textbf{statica} di un processo.

\subsubsection{Risorse}
Iniziamo quindi a vedere quelle che sono le risorse che dobbiamo fornire ai programmi. Il modello che adottiamo è il più diffuso oggigiorno, cioè quello di \textit{Von Neumann}.
Questo modello comprende, a grandi linee:

\begin{itemize}
	\item La \textbf{CPU} o \textit{processore}, che ha il solo scopo di prelevare ed eseguire le istruzioni in maniera sequenziale, alterando il suo flusso come già detto solo nel caso di istruzioni condizionali, o come vedremo nel caso di interruzioni o altre situazioni simili;
	\item La \textbf{memoria principale}, che nell'architettura di Von Neumann contiene sia i dati che il programma in esecuzione, e che deve essere capace di fornire su richiesta alla CPU. 

		Ricordiamo che questa è spesso \textit{volatile}, cioè i suoi contenuti vengono sostanzialmente invalidati allo spengimento della macchina. Potremmo interrogarci sul motivo di tale decisione: principalmente diciamo che le ragioni sono economiche, in quanto mantenere l'informazione per lunghi periodi di tempi è solitamente più costosto e delegato a dispositivi (come i dischi) che offrono risparmi in cambio di grandi tempo di accesso (inusuali sulla memoria principale);
		
	\item Altre \textbf{risorse} che si aggiungono a CPU e memoria, comunque indispensabili per eseguire qualsiasi istruzione. Queste sono:
		\begin{itemize}
			\item I \textbf{dispositivi}, che comprendono ad esempio la memoria di \textit{archiviazione} (il \textbf{disco}) e le \textit{periferiche} di interfaccia con l'utente;
			\item Le risorse \textbf{logiche}, cioè determinate strutture dati in memoria che devono essere fornite in maniera più o meno esclusiva ai processi. Anche gli stessi \textit{file} del \textit{file system} sono risorse logiche.
		\end{itemize}
\end{itemize}

Risulta chiaro come la gestione delle risorse hardware e logiche è fondamentale anche alla \textbf{portabilità} dei programmi, che magari vogliono avere accesso a funzionalità simili su più sistemi operativi (accesso alla tastiera, ai file, ecc...), senza dover necessariamente conoscere l'implementazione interna di tali sistemi operativi.

Abbiamo quindi l'obiettivo di implementare tutte quelle \textbf{interfacce} di cui il programma bisogna per presentare all'utente le sue funzioni. Questo include le interfacce grafiche, audio, ecc... per la realizzazione di ambienti visuali e interattivi nei sistemi moderni.

Dal nostro lato, quello del \textit{sistema}, vorremo che le soluzioni tecniche che adottiamo non impattino in maniera negativa le prestazioni o comunque il funzionamento dei programmi che mandiamo in esecuzione.

\subsubsection{Struttura stratificata del S/O}
La struttura di un sistema operativa può dividersi in più livelli, fra cui:
\begin{itemize}
	\item Il livello \textbf{hardware}, fornito come già detto da risorse come:
		\begin{itemize}
			\item La \textbf{CPU};
			\item La \textbf{memoria principale};
			\item Le \textbf{periferiche}, fra cui \textit{video}, \textit{disco}, \textit{interfacce di rete}, ecc...
		\end{itemize}
		Il livello hardware offre la cosiddettà \textit{intefaccia hardware}, data dalle specifiche secono cui interagiamo con i dispositivi hardware stessi;
	\item Il livello \textbf{sistema operativo} (o \textit{S/O}), che implementa la gestione delle risorse che studieremo nel corso, e offre a sua volta altre risorse logiche. In particolare notiamo:
		\begin{itemize}
			\item Gestione della \textbf{CPU};
			\item Gestione della \textbf{memoria};
			\item Gestione del \textbf{file system} e quindi dei \textit{file};
			\item Gestione dei \textbf{dispositivi} attraverso i \textit{driver}.
		\end{itemize}
		Questo livello offre la sua interfaccia attraverso le \textbf{chiamate di sistema} o \textit{primitive}, che implementano una certa \textbf{API} (\textit{Application Programming Interface}) secondo le quali i programmi utente delegano all'S/O operazioni che non potrebbero normalmente portare avanti da soli (accesso a risorse, schedulazioni temporali, ecc...);
	\item Il livello delle \textbf{applicazioni}, che comprende i programmi utente.
\end{itemize}

Questa gerarchia implica chiaramente che ogni livello non conosce nulla riguardo al livello successivo, ma si preoccupa solo di fornire un'\textit{interfaccia} conforme alle specifiche. A questo punto è compito del livello successivo stesso rispettare l'interfaccia e farne uso per i suoi scopi.

Il programmatore di \textbf{sistema} interagisce con i livelli \textit{hardware} e \textit{S/O}, mentre il programmatore di \textbf{applicazioni} interagisce con i livelli \textit{S/O} e \textit{applicazioni}. 

Compito dell'\textit{API} è quello di generare per i programmatori di applicazioni una macchina \textit{astratta} più semplice da usare, più efficiente e più sicura (cioè realizzare gli obiettivi che ci eravamo posti in 1.1.5).
Ricordiamo che per noi sicurezza significa \textit{modelli} che controllano l'accesso da parte dei processi (altresì \textbf{soggetti}) alla memoria, e più in generale a tutte le risorse sistema (altresì \textbf{oggetti} dei programmi).

\subsubsection{Definizione di S/O}
Iniziamo a definire più nei dettagli cos'è un S/O.
\begin{itemize}
	\item Un S/O è un \textbf{allocatore di risorse}, cioè gestisce \textit{tutte} le risorse, e decide tra richieste conflittuali di accesso a tali risorse (inviate dai vari processi) al fine di garantirne un uso equo ed efficiente.
	\item Un S/O è però anche un \textbf{programma di controllo}, che controlla l'esecuzione dei programmi e lo stato dele risorse per prevenire usi impropri e stati inconsistenti.
\end{itemize}

Ricordiamo che in ogni caso l'unico programma effettivamente in esecuzione in ogni momento sulla macchina reale è il \textbf{kernel}, cioè nucleo, mentre il controllo viene temporaneamente passato fra programmi utente.

\end{document}

\documentclass[a4paper,11pt]{article}
\usepackage[a4paper, margin=8em]{geometry}

% usa i pacchetti per la scrittura in italiano
\usepackage[french,italian]{babel}
\usepackage[T1]{fontenc}
\usepackage[utf8]{inputenc}
\frenchspacing 

% usa i pacchetti per la formattazione matematica
\usepackage{amsmath, amssymb, amsthm, amsfonts}

% usa altri pacchetti
\usepackage{gensymb}
\usepackage{hyperref}
\usepackage{standalone}

\usepackage{colortbl}

\usepackage{xstring}
\usepackage{karnaugh-map}

% imposta il titolo
\title{Appunti Sistemi Operativi}
\author{Luca Seggiani}
\date{2025}

% imposta lo stile
% usa helvetica
\usepackage[scaled]{helvet}
% usa palatino
\usepackage{palatino}
% usa un font monospazio guardabile
\usepackage{lmodern}

\renewcommand{\rmdefault}{ppl}
\renewcommand{\sfdefault}{phv}
\renewcommand{\ttdefault}{lmtt}

% circuiti
\usepackage{circuitikz}
\usetikzlibrary{babel}

% testo cerchiato
\newcommand*\circled[1]{\tikz[baseline=(char.base)]{
            \node[shape=circle,draw,inner sep=2pt] (char) {#1};}}

% disponi il titolo
\makeatletter
\renewcommand{\maketitle} {
	\begin{center} 
		\begin{minipage}[t]{.8\textwidth}
			\textsf{\huge\bfseries \@title} 
		\end{minipage}%
		\begin{minipage}[t]{.2\textwidth}
			\raggedleft \vspace{-1.65em}
			\textsf{\small \@author} \vfill
			\textsf{\small \@date}
		\end{minipage}
		\par
	\end{center}

	\thispagestyle{empty}
	\pagestyle{fancy}
}
\makeatother

% disponi teoremi
\usepackage{tcolorbox}
\newtcolorbox[auto counter, number within=section]{theorem}[2][]{%
	colback=blue!10, 
	colframe=blue!40!black, 
	sharp corners=northwest,
	fonttitle=\sffamily\bfseries, 
	title=Teorema~\thetcbcounter: #2, 
	#1
}

% disponi definizioni
\newtcolorbox[auto counter, number within=section]{definition}[2][]{%
	colback=red!10,
	colframe=red!40!black,
	sharp corners=northwest,
	fonttitle=\sffamily\bfseries,
	title=Definizione~\thetcbcounter: #2,
	#1
}

% disponi codice
\usepackage{listings}
\usepackage[table]{xcolor}

\definecolor{codegreen}{rgb}{0,0.6,0}
\definecolor{codegray}{rgb}{0.5,0.5,0.5}
\definecolor{codepurple}{rgb}{0.58,0,0.82}
\definecolor{backcolour}{rgb}{0.95,0.95,0.92}

\lstdefinestyle{codestyle}{
		backgroundcolor=\color{black!5}, 
		commentstyle=\color{codegreen},
		keywordstyle=\bfseries\color{magenta},
		numberstyle=\sffamily\tiny\color{black!60},
		stringstyle=\color{green!50!black},
		basicstyle=\ttfamily\footnotesize,
		breakatwhitespace=false,         
		breaklines=true,                 
		captionpos=b,                    
		keepspaces=true,                 
		numbers=left,                    
		numbersep=5pt,                  
		showspaces=false,                
		showstringspaces=false,
		showtabs=false,                  
		tabsize=2
}

\lstdefinestyle{shellstyle}{
		backgroundcolor=\color{black!5}, 
		basicstyle=\ttfamily\footnotesize\color{black}, 
		commentstyle=\color{black}, 
		keywordstyle=\color{black},
		numberstyle=\color{black!5},
		stringstyle=\color{black}, 
		showspaces=false,
		showstringspaces=false, 
		showtabs=false, 
		tabsize=2, 
		numbers=none, 
		breaklines=true
}


\lstdefinelanguage{assembler}{ 
  keywords={AAA, AAD, AAM, AAS, ADC, ADCB, ADCW, ADCL, ADD, ADDB, ADDW, ADDL, AND, ANDB, ANDW, ANDL,
        ARPL, BOUND, BSF, BSFL, BSFW, BSR, BSRL, BSRW, BSWAP, BT, BTC, BTCB, BTCW, BTCL, BTR, 
        BTRB, BTRW, BTRL, BTS, BTSB, BTSW, BTSL, CALL, CBW, CDQ, CLC, CLD, CLI, CLTS, CMC, CMP,
        CMPB, CMPW, CMPL, CMPS, CMPSB, CMPSD, CMPSW, CMPXCHG, CMPXCHGB, CMPXCHGW, CMPXCHGL,
        CMPXCHG8B, CPUID, CWDE, DAA, DAS, DEC, DECB, DECW, DECL, DIV, DIVB, DIVW, DIVL, ENTER,
        HLT, IDIV, IDIVB, IDIVW, IDIVL, IMUL, IMULB, IMULW, IMULL, IN, INB, INW, INL, INC, INCB,
        INCW, INCL, INS, INSB, INSD, INSW, INT, INT3, INTO, INVD, INVLPG, IRET, IRETD, JA, JAE,
        JB, JBE, JC, JCXZ, JE, JECXZ, JG, JGE, JL, JLE, JMP, JNA, JNAE, JNB, JNBE, JNC, JNE, JNG,
        JNGE, JNL, JNLE, JNO, JNP, JNS, JNZ, JO, JP, JPE, JPO, JS, JZ, LAHF, LAR, LCALL, LDS,
        LEA, LEAVE, LES, LFS, LGDT, LGS, LIDT, LMSW, LOCK, LODSB, LODSD, LODSW, LOOP, LOOPE,
        LOOPNE, LSL, LSS, LTR, MOV, MOVB, MOVW, MOVL, MOVSB, MOVSD, MOVSW, MOVSX, MOVSXB,
        MOVSXW, MOVSXL, MOVZX, MOVZXB, MOVZXW, MOVZXL, MUL, MULB, MULW, MULL, NEG, NEGB, NEGW,
        NEGL, NOP, NOT, NOTB, NOTW, NOTL, OR, ORB, ORW, ORL, OUT, OUTB, OUTW, OUTL, OUTSB, OUTSD,
        OUTSW, POP, POPL, POPW, POPB, POPA, POPAD, POPF, POPFD, PUSH, PUSHL, PUSHW, PUSHB, PUSHA, 
				PUSHAD, PUSHF, PUSHFD, RCL, RCLB, RCLW, MOVSL, MOVSB, MOVSW, STOSL, STOSB, STOSW, LODSB, LODSW,
				LODSL, INSB, INSW, INSL, OUTSB, OUTSL, OUTSW
        RCLL, RCR, RCRB, RCRW, RCRL, RDMSR, RDPMC, RDTSC, REP, REPE, REPNE, RET, ROL, ROLB, ROLW,
        ROLL, ROR, RORB, RORW, RORL, SAHF, SAL, SALB, SALW, SALL, SAR, SARB, SARW, SARL, SBB,
        SBBB, SBBW, SBBL, SCASB, SCASD, SCASW, SETA, SETAE, SETB, SETBE, SETC, SETE, SETG, SETGE,
        SETL, SETLE, SETNA, SETNAE, SETNB, SETNBE, SETNC, SETNE, SETNG, SETNGE, SETNL, SETNLE,
        SETNO, SETNP, SETNS, SETNZ, SETO, SETP, SETPE, SETPO, SETS, SETZ, SGDT, SHL, SHLB, SHLW,
        SHLL, SHLD, SHR, SHRB, SHRW, SHRL, SHRD, SIDT, SLDT, SMSW, STC, STD, STI, STOSB, STOSD,
        STOSW, STR, SUB, SUBB, SUBW, SUBL, TEST, TESTB, TESTW, TESTL, VERR, VERW, WAIT, WBINVD,
        XADD, XADDB, XADDW, XADDL, XCHG, XCHGB, XCHGW, XCHGL, XLAT, XLATB, XOR, XORB, XORW, XORL},
  keywordstyle=\color{blue}\bfseries,
  ndkeywordstyle=\color{darkgray}\bfseries,
  identifierstyle=\color{black},
  sensitive=false,
  comment=[l]{\#},
  morecomment=[s]{/*}{*/},
  commentstyle=\color{purple}\ttfamily,
  stringstyle=\color{red}\ttfamily,
  morestring=[b]',
  morestring=[b]"
}

\lstset{language=assembler, style=codestyle}

% disponi sezioni
\usepackage{titlesec}

\titleformat{\section}
	{\sffamily\Large\bfseries} 
	{\thesection}{1em}{} 
\titleformat{\subsection}
	{\sffamily\large\bfseries}   
	{\thesubsection}{1em}{} 
\titleformat{\subsubsection}
	{\sffamily\normalsize\bfseries} 
	{\thesubsubsection}{1em}{}

% tikz
\usepackage{tikz}

% float
\usepackage{float}

% grafici
\usepackage{pgfplots}
\pgfplotsset{width=10cm,compat=1.9}

% disponi alberi
\usepackage{forest}

\forestset{
	rectstyle/.style={
		for tree={rectangle,draw,font=\large\sffamily}
	},
	roundstyle/.style={
		for tree={circle,draw,font=\large}
	}
}

% disponi algoritmi
\usepackage{algorithm}
\usepackage{algorithmic}
\makeatletter
\renewcommand{\ALG@name}{Algoritmo}
\makeatother

% disponi numeri di pagina
\usepackage{fancyhdr}
\fancyhf{} 
\fancyfoot[L]{\sffamily{\thepage}}

\makeatletter
\fancyhead[L]{\raisebox{1ex}[0pt][0pt]{\sffamily{\@title \ \@date}}} 
\fancyhead[R]{\raisebox{1ex}[0pt][0pt]{\sffamily{\@author}}}
\makeatother

\begin{document}
% sezione (data)
\section{Lezione del 24-09-25}

% stili pagina
\thispagestyle{empty}
\pagestyle{fancy}

% testo
\subsection{Cenni storici}
Le prime macchine calcolatrici "moderne" nascono durante la seconda guerra mondiale, principalmente per scopi crittografici.

Fu nel periodo del secondo dopoguerra che diverse industrie, principalmente dal settore delle macchine da scrivere e di apparecchiature simili, decisero di sviluppare queste tecnologie per scopi di ricerca e commerciali.

Di pari passo diverse università iniziarono a loro volta a sviluppare architetture e macchine calcolatrici, in questo caso a puro scopo di ricerca. 
Un esempio locale è quello della \textbf{CEP} (\textit{Calcolatrice Elettronica Pisana}), sviluppata dai dipartimenti di matematica e fisica di Pisa (sotto indicazione di Enrico Fermi) per aiutare i ricercatori nei loro calcoli. 

Sempre a Pisa fu l'ingegnere Mario Tchou a lanciare, in collaborazione con Olivetti, il progetto che diventò nel 1959 l'\textbf{Elea 9003}, fra i primi calcolatori a transistor commerciali (di contro la CEP funzionava a valvole termoioniche).

\subsubsection{Sistemi Batch}
In queste prime macchine, anche se la possibilità della multiprogrammazione era disponibile, raramente si parlava di "sistemi operativi" veri e propri.
I primi sistemi operativi nascono quindi per i mainframe degli anni '60, fra cui notiamo gli \textbf{IBM Sistema 360} (e i successivi Sistema 370).

Inizialmente, queste macchine venivano usate in modalità \textbf{batch} (più programmi di più utenti eseguiti in sequenza): i primi S/O nascono appunto per permettere l'uso simultaneo (\textit{time-sharing}) della macchina da parte di più utenti.

In ogni caso, già nei primi sistemi batch monoprogrammati si necessitava di diversi componenti effettivamente assimilabili ad un rudimentale sistema operativo:
\begin{itemize}
	\item Un sistema di programmazione in memoria di massa (all'epoca nastri magnetici);
	\item Una \textit{Job Control Language} (\textbf{JCL}), che esprimeva direttive interpretate da un \textit{Monitor} (antenato delle moderne \textit{shell});
	\item Un \textbf{BIOS} (\textit{Basic Input Output System}), cioè un insieme di routine per l'interazione con le periferiche.
\end{itemize}

L'S/O era quindi composto da Monitor + BIOS, che poteva essere configurato per caricare programmi e mandarli in esecuzione.
In ogni momento in memoria si trovavano comunque il S/O e al più un programma utente.

\subsubsection{Sistemi di spooling}
Il prossimo passo è quello dei sistemi di \textbf{spooling} (\textit{Simultaneous Peripheral Operation On-Line}).
Questi nascono per permettere al programma utente di restare in esecuzione mentre le periferiche (all'epoca molto lente) completano le loro operazioni, bufferrizzando quindi le operazioni di ingresso/uscita.

I sistemi operativi che implementavano lo spooling dovevano quindi arricchirsi per permettere questo tipo di funzionalità.

\subsubsection{Sistemi multiprogrammati}
Arriviamo quindi ai sistemi \textbf{multiprogrammati}, cioè che permettono la gestione contemporanea di più programmi nella memoria principale: per la prima volta oltre al sistema operativo possiamo caricare in memoria più di un singolo programma utente.

I sistemi operativi di questo tipo si dovranno quindi dotare di diverse funzionalità, fra cui \textit{scheduling} dei processi, possibilità di fare \textbf{DMA} (\textit{Direct Memory Access}) sulle periferiche, \textit{preemption} dei programmi in esecuzione, \textit{memoria virtuale} per permettere mappature in memoria localmente costanti per ogni programma, ecc...

\subsubsection{Sistemi time-sharing}
Lo sviluppo di sistemi di tipo multiprogammato è stato favorito dal fatto che i programmi utente che venivano sviluppati erano sempre più \textit{interattivi}, quindi caratterizzati da fasi temporali distinte:
\begin{itemize}
	\item \textbf{CPU-Burst}, dove il processore lavorava effettivamente sui dati;
	\item \textbf{I/O-Burst}, dove il processore attendeva operazioni I/O dalle periferiche, magari fornendosi del DMA.
\end{itemize}

Ci spostiamo quindi da un paradigma di esecuzione \textit{sequenziale} ad un paradigma \textit{multi-tasking}, dove il sistema operativo assegna ciclicamente istanti temporali (\textit{quantum}) ai processi in esecuzione.

Il vantaggio dell'esecuzione multitasking è di poter avvicinare fra di loro i CPU-Burst, spostando il controllo della CPU da un processo all'altro quando si incorre in un I/O-Burst.

Per quanto ci riguarda, quindi, la tecnica del \textbf{time-sharing} non è che un modo per implementare il \textit{multi-tasking}, cioè un caso particolare della \textit{multiprogrammazione}, caratterizzato da processi in memoria che vengono eseguiti (o almeno hanno l'illusione di essere eseguiti) contemporaneamente. 
Ricordiamo che l'esistenza di più processi in memoria era di per sé caratteristica del sistema multiprogrammato.

L'idea di sviluppare diversi e sofisticati algoritmi di \textit{scheduling} viene proprio dalla necessità di dover mantenere la CPU in piena attività, cioé eseguire più CPU-Burst possibile, scegliendo in maniera intelligente quali processi mandare in esecuzione (equivalentemente, a quali processi assegnare i quantum temporali).

Notiamo che il tempo che la CPU passa a realizzare lo scheduling e i cambi di contesto rappresenta effettivamente \textbf{overhead} per il sistema, cioè tempo non passato ad eseguire programmi utente, ma in qualche modo "sprecato" in altri modi.
Questo overhead è giustificato solo nel caso in cui le virtualizzazioni che consente permettono una velocizzazione considerevole della macchina.

\subsubsection{Sistemi in tempo reale}
La storia dei sistemi operativi ha un'interessante tangente nei cosiddetti sistemi \textbf{real-time} (\textit{in tempo reale}).
Questi sono sistemi dove lo scheduling è \textit{deterministico} e il tempo impiegato ad eseguire un dato processo può quindi essere stabilito prima che questo venga lanciato.

Sistemi di questo tipo sono utili nel caso di calcolatori che interagiscono con \textit{ambienti operativi} reali attraverso \textbf{sensori} ed \textbf{attuatori}, dove la precisione temporale con cui vengono eseguite certe operazioni è effettivamente importante alla funzione della macchina.

In particolare notiamo due paradigmi possibili per i sistemi real-time:
\begin{itemize}
	\item \textbf{Soft} real-time, che non assicurano ma si impegnano a mantenere le specifiche sopra descritte;
	\item \textbf{Hard} real-time, il cui funzionamento ha come priorità imprescindibile le specifiche sopra descritte. 
\end{itemize}

\end{document}

\documentclass[a4paper,11pt]{article}
\usepackage[a4paper, margin=8em]{geometry}

% usa i pacchetti per la scrittura in italiano
\usepackage[french,italian]{babel}
\usepackage[T1]{fontenc}
\usepackage[utf8]{inputenc}
\frenchspacing 

% usa i pacchetti per la formattazione matematica
\usepackage{amsmath, amssymb, amsthm, amsfonts}

% usa altri pacchetti
\usepackage{gensymb}
\usepackage{hyperref}
\usepackage{standalone}

\usepackage{colortbl}

\usepackage{xstring}
\usepackage{karnaugh-map}

% imposta il titolo
\title{Appunti Sistemi Operativi}
\author{Luca Seggiani}
\date{2025}

% imposta lo stile
% usa helvetica
\usepackage[scaled]{helvet}
% usa palatino
\usepackage{palatino}
% usa un font monospazio guardabile
\usepackage{lmodern}

\renewcommand{\rmdefault}{ppl}
\renewcommand{\sfdefault}{phv}
\renewcommand{\ttdefault}{lmtt}

% circuiti
\usepackage{circuitikz}
\usetikzlibrary{babel}

% testo cerchiato
\newcommand*\circled[1]{\tikz[baseline=(char.base)]{
            \node[shape=circle,draw,inner sep=2pt] (char) {#1};}}

% disponi il titolo
\makeatletter
\renewcommand{\maketitle} {
	\begin{center} 
		\begin{minipage}[t]{.8\textwidth}
			\textsf{\huge\bfseries \@title} 
		\end{minipage}%
		\begin{minipage}[t]{.2\textwidth}
			\raggedleft \vspace{-1.65em}
			\textsf{\small \@author} \vfill
			\textsf{\small \@date}
		\end{minipage}
		\par
	\end{center}

	\thispagestyle{empty}
	\pagestyle{fancy}
}
\makeatother

% disponi teoremi
\usepackage{tcolorbox}
\newtcolorbox[auto counter, number within=section]{theorem}[2][]{%
	colback=blue!10, 
	colframe=blue!40!black, 
	sharp corners=northwest,
	fonttitle=\sffamily\bfseries, 
	title=Teorema~\thetcbcounter: #2, 
	#1
}

% disponi definizioni
\newtcolorbox[auto counter, number within=section]{definition}[2][]{%
	colback=red!10,
	colframe=red!40!black,
	sharp corners=northwest,
	fonttitle=\sffamily\bfseries,
	title=Definizione~\thetcbcounter: #2,
	#1
}

% disponi codice
\usepackage{listings}
\usepackage[table]{xcolor}

\definecolor{codegreen}{rgb}{0,0.6,0}
\definecolor{codegray}{rgb}{0.5,0.5,0.5}
\definecolor{codepurple}{rgb}{0.58,0,0.82}
\definecolor{backcolour}{rgb}{0.95,0.95,0.92}

\lstdefinestyle{codestyle}{
		backgroundcolor=\color{black!5}, 
		commentstyle=\color{codegreen},
		keywordstyle=\bfseries\color{magenta},
		numberstyle=\sffamily\tiny\color{black!60},
		stringstyle=\color{green!50!black},
		basicstyle=\ttfamily\footnotesize,
		breakatwhitespace=false,         
		breaklines=true,                 
		captionpos=b,                    
		keepspaces=true,                 
		numbers=left,                    
		numbersep=5pt,                  
		showspaces=false,                
		showstringspaces=false,
		showtabs=false,                  
		tabsize=2
}

\lstdefinestyle{shellstyle}{
		backgroundcolor=\color{black!5}, 
		basicstyle=\ttfamily\footnotesize\color{black}, 
		commentstyle=\color{black}, 
		keywordstyle=\color{black},
		numberstyle=\color{black!5},
		stringstyle=\color{black}, 
		showspaces=false,
		showstringspaces=false, 
		showtabs=false, 
		tabsize=2, 
		numbers=none, 
		breaklines=true
}


\lstdefinelanguage{assembler}{ 
  keywords={AAA, AAD, AAM, AAS, ADC, ADCB, ADCW, ADCL, ADD, ADDB, ADDW, ADDL, AND, ANDB, ANDW, ANDL,
        ARPL, BOUND, BSF, BSFL, BSFW, BSR, BSRL, BSRW, BSWAP, BT, BTC, BTCB, BTCW, BTCL, BTR, 
        BTRB, BTRW, BTRL, BTS, BTSB, BTSW, BTSL, CALL, CBW, CDQ, CLC, CLD, CLI, CLTS, CMC, CMP,
        CMPB, CMPW, CMPL, CMPS, CMPSB, CMPSD, CMPSW, CMPXCHG, CMPXCHGB, CMPXCHGW, CMPXCHGL,
        CMPXCHG8B, CPUID, CWDE, DAA, DAS, DEC, DECB, DECW, DECL, DIV, DIVB, DIVW, DIVL, ENTER,
        HLT, IDIV, IDIVB, IDIVW, IDIVL, IMUL, IMULB, IMULW, IMULL, IN, INB, INW, INL, INC, INCB,
        INCW, INCL, INS, INSB, INSD, INSW, INT, INT3, INTO, INVD, INVLPG, IRET, IRETD, JA, JAE,
        JB, JBE, JC, JCXZ, JE, JECXZ, JG, JGE, JL, JLE, JMP, JNA, JNAE, JNB, JNBE, JNC, JNE, JNG,
        JNGE, JNL, JNLE, JNO, JNP, JNS, JNZ, JO, JP, JPE, JPO, JS, JZ, LAHF, LAR, LCALL, LDS,
        LEA, LEAVE, LES, LFS, LGDT, LGS, LIDT, LMSW, LOCK, LODSB, LODSD, LODSW, LOOP, LOOPE,
        LOOPNE, LSL, LSS, LTR, MOV, MOVB, MOVW, MOVL, MOVSB, MOVSD, MOVSW, MOVSX, MOVSXB,
        MOVSXW, MOVSXL, MOVZX, MOVZXB, MOVZXW, MOVZXL, MUL, MULB, MULW, MULL, NEG, NEGB, NEGW,
        NEGL, NOP, NOT, NOTB, NOTW, NOTL, OR, ORB, ORW, ORL, OUT, OUTB, OUTW, OUTL, OUTSB, OUTSD,
        OUTSW, POP, POPL, POPW, POPB, POPA, POPAD, POPF, POPFD, PUSH, PUSHL, PUSHW, PUSHB, PUSHA, 
				PUSHAD, PUSHF, PUSHFD, RCL, RCLB, RCLW, MOVSL, MOVSB, MOVSW, STOSL, STOSB, STOSW, LODSB, LODSW,
				LODSL, INSB, INSW, INSL, OUTSB, OUTSL, OUTSW
        RCLL, RCR, RCRB, RCRW, RCRL, RDMSR, RDPMC, RDTSC, REP, REPE, REPNE, RET, ROL, ROLB, ROLW,
        ROLL, ROR, RORB, RORW, RORL, SAHF, SAL, SALB, SALW, SALL, SAR, SARB, SARW, SARL, SBB,
        SBBB, SBBW, SBBL, SCASB, SCASD, SCASW, SETA, SETAE, SETB, SETBE, SETC, SETE, SETG, SETGE,
        SETL, SETLE, SETNA, SETNAE, SETNB, SETNBE, SETNC, SETNE, SETNG, SETNGE, SETNL, SETNLE,
        SETNO, SETNP, SETNS, SETNZ, SETO, SETP, SETPE, SETPO, SETS, SETZ, SGDT, SHL, SHLB, SHLW,
        SHLL, SHLD, SHR, SHRB, SHRW, SHRL, SHRD, SIDT, SLDT, SMSW, STC, STD, STI, STOSB, STOSD,
        STOSW, STR, SUB, SUBB, SUBW, SUBL, TEST, TESTB, TESTW, TESTL, VERR, VERW, WAIT, WBINVD,
        XADD, XADDB, XADDW, XADDL, XCHG, XCHGB, XCHGW, XCHGL, XLAT, XLATB, XOR, XORB, XORW, XORL},
  keywordstyle=\color{blue}\bfseries,
  ndkeywordstyle=\color{darkgray}\bfseries,
  identifierstyle=\color{black},
  sensitive=false,
  comment=[l]{\#},
  morecomment=[s]{/*}{*/},
  commentstyle=\color{purple}\ttfamily,
  stringstyle=\color{red}\ttfamily,
  morestring=[b]',
  morestring=[b]"
}

\lstset{language=assembler, style=codestyle}

% disponi sezioni
\usepackage{titlesec}

\titleformat{\section}
	{\sffamily\Large\bfseries} 
	{\thesection}{1em}{} 
\titleformat{\subsection}
	{\sffamily\large\bfseries}   
	{\thesubsection}{1em}{} 
\titleformat{\subsubsection}
	{\sffamily\normalsize\bfseries} 
	{\thesubsubsection}{1em}{}

% tikz
\usepackage{tikz}

% float
\usepackage{float}

% grafici
\usepackage{pgfplots}
\pgfplotsset{width=10cm,compat=1.9}

% disponi alberi
\usepackage{forest}

\forestset{
	rectstyle/.style={
		for tree={rectangle,draw,font=\large\sffamily}
	},
	roundstyle/.style={
		for tree={circle,draw,font=\large}
	}
}

% disponi algoritmi
\usepackage{algorithm}
\usepackage{algorithmic}
\makeatletter
\renewcommand{\ALG@name}{Algoritmo}
\makeatother

% disponi numeri di pagina
\usepackage{fancyhdr}
\fancyhf{} 
\fancyfoot[L]{\sffamily{\thepage}}

\makeatletter
\fancyhead[L]{\raisebox{1ex}[0pt][0pt]{\sffamily{\@title \ \@date}}} 
\fancyhead[R]{\raisebox{1ex}[0pt][0pt]{\sffamily{\@author}}}
\makeatother

\begin{document}
% sezione (data)
\section{Lezione del 25-09-25}

% stili pagina
\thispagestyle{empty}
\pagestyle{fancy}

% testo
\subsection{Richiami architetturali}
Riprendiamo alcuni aspetti architetturali di un sistema di elaborazione.
L'architettura che consideriamo è quella di \textit{Von Neumann}, modello ancora oggi in uso e composto da:
\begin{itemize}
	\item La \textbf{CPU} (\textit{Central Processing Unit}) o come abbiamo già detto \textit{processore}. Rappresenta un circuito piuttosto complesso che ha però l'unica funzione di \textit{esecutore di istruzioni}.

		Le istruzioni che questa esegue possono essere di tipo \textbf{CISC} (\textit{Complex Instruction Set}), come ad esempio nell'architettura x86, o di tipo \textbf{RISC} (\textit{Reduced Instruction Set}), come ad esempio nell'architettura ARM. Ricordiamo comunque che nelle moderne implementazioni dell'x86 si traduce comunque in un instruction set RISC a livello architetturale per questioni di ottimizzazione.

		Si può infatti dire che è inutile avere molte e complesse istruzioni (CISC) che richiedono molti cicli di clock, quandi si possono avere poche e semplici istruzioni (RISC) che ne richiedono pochi: eventuali istruzioni più complesse potranno essere implementate come \textit{subroutine} che usano più istruzioni semplici.

		Ricordiamo quindi che la CPU si limita ad eseguire istruzioni, e non conosce (non memorizza) il programma.
		La poca memoria che ha a disposizione (sotto forma di \textit{registri}) viene usata per mantenere i dati che sta elaborando;
	\item La \textbf{RAM} o \textit{memoria centrale}, o ancora come abbiamo visto \textit{memoria principale}. Questa ha il compito di memorizzare \textit{dati} e \textit{programma} (questo il fulcro dell'architettura di Von Neumann) e di renderli disponibili alla CPU e, come vedremo, anche ad altri dispositivi.

		Abbiamo visto che è una memoria \textit{volatile}, quindi che si mantiene solo finché il calcolatore è acceso, e che è una memoria ad \textit{accesso diretto}, cioè si può accedere a qualsiasi locazione in tempo costante (a differenza di memorie di tipo \textit{sequenziale}, ecc...).

		Le operazioni che possiamo svolgere sulla memoria sono \textit{letture} e \textit{scritture} su locazioni di memoria.
		Nelle memorie moderne le letture sono \textit{non distruttive}, mentre le scritture (chiaramente) lo sono.

	\item Qualche tipo di complesso di \textbf{I/O}. Questo comprende periferiche come \textit{tastiera}, \textit{porte seriali/parallele}, \textit{interfacce di rete}, ecc...

		Un dispositivo particolare che si trova nello spazio di I/O è il \textbf{disco} o \textit{memoria secondaria}, a differenza della principale \textit{persistente}, e usata per l'archiviazione di dati a lungo termine.
		Chiaramente, il tradeoff in questo caso è in termini di tempo (i dischi, anche allo stato solido, sono molto più lenti in tempo di accesso della RAM).

	\item Un \textbf{bus}, o \textit{rete di interconnessione}, che permette a questi componenti di comunicare fra di loro.

		Questa comunicazione dovrà essere \textbf{bidirezionale}, in quanto ad esempio la CPU deve sia leggere che scrivere dalla RAM: abbiamo visto come bus di questo tipo possono essere implementati sfruttando la logica a 3 stati.

		Sperabilmente un bus dovrà contenere un numero consistente di linee.
		Torniamo all'esempio della CPU che legge in memoria: avremo bisogno di specificare l'\textit{indirizzo} della locazione che vogliamo leggere, e vorremo vederci tornare una o più \textit{parole} (cioè i dati che ci interessano) dalla memoria.
		Il modo più veloce per effettuare questa operazione è fornirsi di abbastanza linee per specificare sia gli indirizzi che i dati in \textbf{parallelo}: un bus \textit{seriale} si dimostrerebbe molto più lento.

		A livello logico dobbiamo dire anche che c'è bisogno di un \textbf{protocollo}, o comunque una qualche \textit{politica} di gestione del bus.
	\begin{itemize}
		\item Ad esempio, la politica più semplice è quella dove la CPU è l'unica che può iniziare una transazione sul bus: questa è la classica configurazione \textit{master-slave} dove la CPU rappresenta il \textit{master} e memoria e I/O rappresentano gli \textit{slave};
		\item Esistono però situazioni dove potremmo volere che i dispositivi (ad esempio il disco) scrivano in memoria, o viceversa sia la memoria a scrivere sui dispositivi. Questo è effettivamente il caso del \textit{DMA}.
			Avere un bus che supporta più iniziatori di transazioni richiede necessariamente un protocollo che stabilisca chiaramente chi può iniziare in quale momento una data transazione.
	\end{itemize}

	Le transazioni avvengono chiaramente in fasi, di cui ne individuiamo almeno 3 nel caso più semplice (singolo master, più slave):
	\begin{enumerate}
		\item Una prima fase di richiesta della transazione da parte dell'\textit{iniziatore};
		\item Una fase di attesa da parte dell'iniziatore del responso dell'\textit{obiettivo};
		\item Una fase dove l'operazione viene effettivamente eseguita, in un determinato lasso di tempo.
	\end{enumerate}

	Nel caso di più master, abbiamo bisogno di meccanismi più sofisticati che implementino \textbf{mutua esclusione} e \textbf{sincronizzazione} delle risorse a cui i più iniziatori potrebbero voler accedere.
	Questo è vero sia a livello \textit{logico} (su risorse logiche o comunque gestite dal S/O) che \textit{elettrico} (2 o più componenti non pilotino mai le stesse linee contemporaneamente, pena fili bruciati).

\end{itemize}

Facciamo quindi una considerazione su come organizzare lo spazio di memoria e lo spazio dedicato ai registri delle periferiche.
Esistono due configurazioni principali:
\begin{itemize}
	\item \textbf{Memory-mapped I/O}: disponiamo i registri di I/O direttamente nello spazio di memoria, usando gli stessi indirizzi per indirizzare sia la memoria che i dispositivi;
	\item \textbf{Port-mapped I/O}: sfruttiamo due spazi, lo \textit{spazio di memoria} e lo \textit{spazio di I/O}, che mantengono separati i due tipi di informazione. Questo può essere fatto agilmente includendo un bit di selezione di spazio nel bus, ed è la soluzione adottata dall'architettura x86. 
\end{itemize}

\subsection{CPU}
Vediamo nel dettaglio il primo componente, cioè la CPU. 

\subsubsection{Cicli CPU}
Il funzionamento della CPU avviene in maniera \textbf{ciclica}: cogliamo più fasi che si ripetono nel tempo da quando questa viene accesa (reset) fino a quando viene spenta.
\begin{enumerate}
	\item \textbf{Prelievo} o \textit{fetch}: si legge la prossima istruzione in memoria, puntata dall'\textbf{IP} o \textbf{PC} (\textit{Instruction Pointer} o \textit{Program Counter}), e la si porta in un qualche registro interno al processore, pronta ad essere eseguita;
	\item \textbf{Decodifica} o \textit{decode}: si interpreta il significato dell'istruzione, cioè si individua qual'è effettivamente l'istruzione che dobbiamo eseguire, e si portano all'interno di registri gli eventuali \textit{operandi sorgente} o gli indirizzi degli \textit{operandi destinazione};
	\item \textbf{Esecuzione} o \textit{execute}: si esegue effettivamente l'istruzione, direttamente attraverso la rete di controllo della CPU o sfruttando una o più \textbf{ALU} (\textit{Arithmetic and Logic Unit}). 

		Successivamente, il risultato viene (se necessario) riscritto in memoria attraverso un'operazione di \textit{write-back}. Questa fase viene a volte considerata come a sé stante (ad esempio nelle pipeline delle architetture RISC).
\end{enumerate}

\subsubsection{Registri CPU}
La CPU è dotata di una sua memoria interna formata da locazioni di memoria dette \textbf{registri}.
Questi si dividono in registri \textbf{generali}, riservati alle elaborazioni, e \textbf{di stato}, riservati a compiti speciali.

\par\medskip
\textbf{\textsf{Registri generali}} \\
Consideriamo un set estremamente generico di registri:
\begin{itemize}
	\item \textbf{AX}, \textbf{BX}, \textbf{CX} e \textbf{DX} sono i classici registri programmatore a uso generale;
	\item \textbf{ESP} è utilizzato per indirizzare la \textbf{pila} o \textbf{stack}, ovvero una parte di memoria con disciplina LIFO che serve a gestire sottoprogrammi.
\end{itemize}

\par\medskip
\textbf{\textsf{Registri di stato}} \\
Ricordiamo due registri di stato:
\begin{itemize}
	\item  L'\textbf{IP} o \textbf{PC} (\textit{instruction pointer} o \textit{program counter}).
Viene usato per contenere l'indirizzo della locazione dalla quale sarà prelevata la prossima istruzione da eseguire.
Il contenuto dell'EIP è fissato al reset iniziale, e impostato sulla prima istruzione da eseguire.

\item L'\textbf{F} (registro dei \textit{flag}).
Consiste di una serie di elementi binari detti \textbf{flag}, fra cui ricordiamo:
\begin{itemize}
	\item \textbf{OF:} flag di overflow (traboccamento) delle operazioni aritmetiche, si imposta se l'ultima operazioni, presi gli operandi come interi, ha prodotto un risultato non rappresentabile su $n$ bit;
	\item \textbf{SF:} flag di segno, impostato quando l'ultima operazione restituisce un complemento a 2 con MSB $= 1$ (ergo negativo);
	\item \textbf{ZF:} flag zero, che viene impostato quando l'ultima operazione restituisce qualcosa di nullo;
	\item \textbf{CF:} flag di carry (riporto), che viene impostato quando l'ultima operazione richiede un riporto o un prestito, ergo presi gli operandi come naturali il risultato non è rappresentabile su $n$ bit;
	\item \textbf{IF:} flag di interruzioni attivate, quando è attivo il processore risponde alle interruzioni (che approfondiremo in seguito). 
\end{itemize}
\end{itemize}

Al reset i flag visti finora sono impostati a 0.

\subsubsection{Instruction set}
Consideriamo un set di istruzioni estremamente basilare.
Innanzitutto, possiamo dividere le istruzioni in \textbf{operative} e \textbf{di controllo}.
Possiamo quindi fare ulteriori suddivisioni al'interno di queste categorie:
\begin{itemize}
	\item \textbf{Operative:}
		\begin{itemize}
			\item Di trasferimento;
			\item Aritmetiche;
			\item Di traslazione/rotazione:
			\item Logiche.
		\end{itemize}
	\item \textbf{Di controllo:}
		\begin{itemize}
			\item Di salto;
			\item Di gestione di sottoprogrammi.
		\end{itemize}
\end{itemize}

# discorso CISC

\par\medskip

Per informazioni più approfondite sulla struttura generale del processore considerato si rimanda ai testi specializzati o agli appunti in \url{https://raw.githubusercontent.com/seggiani-luca/appunti-rl/34228f66db395637bd1824d04f3130b977cc0ce4/master/master.pdf}.

\subsection{RAM}
Approfondiamo quindi il discorso della RAM.
Nel sistema considerato inseriremo un elemento di \textbf{cache}, nello specifico fra la CPU e il bus (da cui si accede alla RAM).
Il funzionamento della cache è dettagliato in \url{https://raw.githubusercontent.com/seggiani-luca/appunti-ce/638d3abf2e1d473632b575401582203c3b113c82/master/master.pdf}, e per quanto ci riguarda possiamo dire che funge da unità di "\textit{memoizzazion}" dei dati (quando vengono richiesti), più veloce della RAM.

\end{document}

\end{document}