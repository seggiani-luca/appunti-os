\documentclass[a4paper,11pt]{article}
\usepackage[a4paper, margin=8em]{geometry}

% usa i pacchetti per la scrittura in italiano
\usepackage[french,italian]{babel}
\usepackage[T1]{fontenc}
\usepackage[utf8]{inputenc}
\frenchspacing 

% usa i pacchetti per la formattazione matematica
\usepackage{amsmath, amssymb, amsthm, amsfonts}

% usa altri pacchetti
\usepackage{gensymb}
\usepackage{hyperref}
\usepackage{standalone}

\usepackage{colortbl}

\usepackage{xstring}
\usepackage{karnaugh-map}

% imposta il titolo
\title{Appunti Sistemi Operativi}
\author{Luca Seggiani}
\date{2025}

% imposta lo stile
% usa helvetica
\usepackage[scaled]{helvet}
% usa palatino
\usepackage{palatino}
% usa un font monospazio guardabile
\usepackage{lmodern}

\renewcommand{\rmdefault}{ppl}
\renewcommand{\sfdefault}{phv}
\renewcommand{\ttdefault}{lmtt}

% circuiti
\usepackage{circuitikz}
\usetikzlibrary{babel}

% testo cerchiato
\newcommand*\circled[1]{\tikz[baseline=(char.base)]{
            \node[shape=circle,draw,inner sep=2pt] (char) {#1};}}

% disponi il titolo
\makeatletter
\renewcommand{\maketitle} {
	\begin{center} 
		\begin{minipage}[t]{.8\textwidth}
			\textsf{\huge\bfseries \@title} 
		\end{minipage}%
		\begin{minipage}[t]{.2\textwidth}
			\raggedleft \vspace{-1.65em}
			\textsf{\small \@author} \vfill
			\textsf{\small \@date}
		\end{minipage}
		\par
	\end{center}

	\thispagestyle{empty}
	\pagestyle{fancy}
}
\makeatother

% disponi teoremi
\usepackage{tcolorbox}
\newtcolorbox[auto counter, number within=section]{theorem}[2][]{%
	colback=blue!10, 
	colframe=blue!40!black, 
	sharp corners=northwest,
	fonttitle=\sffamily\bfseries, 
	title=Teorema~\thetcbcounter: #2, 
	#1
}

% disponi definizioni
\newtcolorbox[auto counter, number within=section]{definition}[2][]{%
	colback=red!10,
	colframe=red!40!black,
	sharp corners=northwest,
	fonttitle=\sffamily\bfseries,
	title=Definizione~\thetcbcounter: #2,
	#1
}

% disponi codice
\usepackage{listings}
\usepackage[table]{xcolor}

\definecolor{codegreen}{rgb}{0,0.6,0}
\definecolor{codegray}{rgb}{0.5,0.5,0.5}
\definecolor{codepurple}{rgb}{0.58,0,0.82}
\definecolor{backcolour}{rgb}{0.95,0.95,0.92}

\lstdefinestyle{codestyle}{
		backgroundcolor=\color{black!5}, 
		commentstyle=\color{codegreen},
		keywordstyle=\bfseries\color{magenta},
		numberstyle=\sffamily\tiny\color{black!60},
		stringstyle=\color{green!50!black},
		basicstyle=\ttfamily\footnotesize,
		breakatwhitespace=false,         
		breaklines=true,                 
		captionpos=b,                    
		keepspaces=true,                 
		numbers=left,                    
		numbersep=5pt,                  
		showspaces=false,                
		showstringspaces=false,
		showtabs=false,                  
		tabsize=2
}

\lstdefinestyle{shellstyle}{
		backgroundcolor=\color{black!5}, 
		basicstyle=\ttfamily\footnotesize\color{black}, 
		commentstyle=\color{black}, 
		keywordstyle=\color{black},
		numberstyle=\color{black!5},
		stringstyle=\color{black}, 
		showspaces=false,
		showstringspaces=false, 
		showtabs=false, 
		tabsize=2, 
		numbers=none, 
		breaklines=true
}


\lstdefinelanguage{assembler}{ 
  keywords={AAA, AAD, AAM, AAS, ADC, ADCB, ADCW, ADCL, ADD, ADDB, ADDW, ADDL, AND, ANDB, ANDW, ANDL,
        ARPL, BOUND, BSF, BSFL, BSFW, BSR, BSRL, BSRW, BSWAP, BT, BTC, BTCB, BTCW, BTCL, BTR, 
        BTRB, BTRW, BTRL, BTS, BTSB, BTSW, BTSL, CALL, CBW, CDQ, CLC, CLD, CLI, CLTS, CMC, CMP,
        CMPB, CMPW, CMPL, CMPS, CMPSB, CMPSD, CMPSW, CMPXCHG, CMPXCHGB, CMPXCHGW, CMPXCHGL,
        CMPXCHG8B, CPUID, CWDE, DAA, DAS, DEC, DECB, DECW, DECL, DIV, DIVB, DIVW, DIVL, ENTER,
        HLT, IDIV, IDIVB, IDIVW, IDIVL, IMUL, IMULB, IMULW, IMULL, IN, INB, INW, INL, INC, INCB,
        INCW, INCL, INS, INSB, INSD, INSW, INT, INT3, INTO, INVD, INVLPG, IRET, IRETD, JA, JAE,
        JB, JBE, JC, JCXZ, JE, JECXZ, JG, JGE, JL, JLE, JMP, JNA, JNAE, JNB, JNBE, JNC, JNE, JNG,
        JNGE, JNL, JNLE, JNO, JNP, JNS, JNZ, JO, JP, JPE, JPO, JS, JZ, LAHF, LAR, LCALL, LDS,
        LEA, LEAVE, LES, LFS, LGDT, LGS, LIDT, LMSW, LOCK, LODSB, LODSD, LODSW, LOOP, LOOPE,
        LOOPNE, LSL, LSS, LTR, MOV, MOVB, MOVW, MOVL, MOVSB, MOVSD, MOVSW, MOVSX, MOVSXB,
        MOVSXW, MOVSXL, MOVZX, MOVZXB, MOVZXW, MOVZXL, MUL, MULB, MULW, MULL, NEG, NEGB, NEGW,
        NEGL, NOP, NOT, NOTB, NOTW, NOTL, OR, ORB, ORW, ORL, OUT, OUTB, OUTW, OUTL, OUTSB, OUTSD,
        OUTSW, POP, POPL, POPW, POPB, POPA, POPAD, POPF, POPFD, PUSH, PUSHL, PUSHW, PUSHB, PUSHA, 
				PUSHAD, PUSHF, PUSHFD, RCL, RCLB, RCLW, MOVSL, MOVSB, MOVSW, STOSL, STOSB, STOSW, LODSB, LODSW,
				LODSL, INSB, INSW, INSL, OUTSB, OUTSL, OUTSW
        RCLL, RCR, RCRB, RCRW, RCRL, RDMSR, RDPMC, RDTSC, REP, REPE, REPNE, RET, ROL, ROLB, ROLW,
        ROLL, ROR, RORB, RORW, RORL, SAHF, SAL, SALB, SALW, SALL, SAR, SARB, SARW, SARL, SBB,
        SBBB, SBBW, SBBL, SCASB, SCASD, SCASW, SETA, SETAE, SETB, SETBE, SETC, SETE, SETG, SETGE,
        SETL, SETLE, SETNA, SETNAE, SETNB, SETNBE, SETNC, SETNE, SETNG, SETNGE, SETNL, SETNLE,
        SETNO, SETNP, SETNS, SETNZ, SETO, SETP, SETPE, SETPO, SETS, SETZ, SGDT, SHL, SHLB, SHLW,
        SHLL, SHLD, SHR, SHRB, SHRW, SHRL, SHRD, SIDT, SLDT, SMSW, STC, STD, STI, STOSB, STOSD,
        STOSW, STR, SUB, SUBB, SUBW, SUBL, TEST, TESTB, TESTW, TESTL, VERR, VERW, WAIT, WBINVD,
        XADD, XADDB, XADDW, XADDL, XCHG, XCHGB, XCHGW, XCHGL, XLAT, XLATB, XOR, XORB, XORW, XORL},
  keywordstyle=\color{blue}\bfseries,
  ndkeywordstyle=\color{darkgray}\bfseries,
  identifierstyle=\color{black},
  sensitive=false,
  comment=[l]{\#},
  morecomment=[s]{/*}{*/},
  commentstyle=\color{purple}\ttfamily,
  stringstyle=\color{red}\ttfamily,
  morestring=[b]',
  morestring=[b]"
}

\lstset{language=assembler, style=codestyle}

% disponi sezioni
\usepackage{titlesec}

\titleformat{\section}
	{\sffamily\Large\bfseries} 
	{\thesection}{1em}{} 
\titleformat{\subsection}
	{\sffamily\large\bfseries}   
	{\thesubsection}{1em}{} 
\titleformat{\subsubsection}
	{\sffamily\normalsize\bfseries} 
	{\thesubsubsection}{1em}{}

% tikz
\usepackage{tikz}

% float
\usepackage{float}

% grafici
\usepackage{pgfplots}
\pgfplotsset{width=10cm,compat=1.9}

% disponi alberi
\usepackage{forest}

\forestset{
	rectstyle/.style={
		for tree={rectangle,draw,font=\large\sffamily}
	},
	roundstyle/.style={
		for tree={circle,draw,font=\large}
	}
}

% disponi algoritmi
\usepackage{algorithm}
\usepackage{algorithmic}
\makeatletter
\renewcommand{\ALG@name}{Algoritmo}
\makeatother

% disponi numeri di pagina
\usepackage{fancyhdr}
\fancyhf{} 
\fancyfoot[L]{\sffamily{\thepage}}

\makeatletter
\fancyhead[L]{\raisebox{1ex}[0pt][0pt]{\sffamily{\@title \ \@date}}} 
\fancyhead[R]{\raisebox{1ex}[0pt][0pt]{\sffamily{\@author}}}
\makeatother

\begin{document}
% sezione (data)
\section{Lezione del 29-10-25}

% stili pagina
\thispagestyle{empty}
\pagestyle{fancy}

% testo
\subsection{Implementazione di un monitor}
Veniamo quindi a come si può effettivamente implementare un monitor come descritto nella scorsa lezione.

Avere più funzioni in mutua esclusione significa effettivamente usare un semaforo di mutex \lstinline|sem mutex = 1|, e avere il seguente prologo ed epilogo di funzione per ogni funzione interna al monitor:
\begin{lstlisting}[language=C++, style=codestyle]	
func_monitor() {
	// prologo
	wait(mutex);

	// corpo func

	// epilogo
	signal(mutex);
}
\end{lstlisting}

Il problema diventa quindi la gestione delle variabili di condizione:
\begin{itemize}
	\item Ricordiamo che \lstinline|x.wait()| vuole che il processo attuale si sospenda;
	\item \lstinline|x.signal()| potrebbe invece bloccare il processo e passare ad un altri (\textit{signal and wait}) oppure continuare col processo corrente (\textit{signal and continue}). Noi, come anticipato in 13.3.1, useremo la prima politica.  
\end{itemize}

Avremo quindi un semaforo inizializato a zero su ogni variabile di condizione (ad esempio \lstinline|sem x_sem = 0|) per il blocco, e un contatore dei processi bloccati sulla variabile (ad esempio \lstinline|int x_count = 0|).
\begin{itemize}
	\item A questo punto la \lstinline|x.wait()| sarà:
\begin{lstlisting}[language=C++, style=codestyle]	
x.wait() {
	x_count++;

	signal(mutex); // devo sbloccare il mutex 
	wait(x_sem);
}
\end{lstlisting}

Il problema è che facciamo una \lstinline|signal(mutex)|, quando in verità vorremmo segnalare di proseguire ai processi già interni al monitor.
Modifichiamo allora il moniotr, introducendo un semaforo \lstinline|sem next = 0| e un contatore \lstinline|int next_count = 0| per i processi \textit{"bloccati"} al suo interno.

Prologo ed epilogo saranno allora:
\begin{lstlisting}[language=C++, style=codestyle]	
func_monitor() {
	// prologo
	wait(mutex);

	// corpo func

	// epilogo
	if(next_count > 0) {
		signal(next); // prima fai uscire i processi interni
	} else {
		signal(mutex); // poi apri il monitor ad altri
	}
}
\end{lstlisting}


A questo punto il codice della \lstinline|x.wait()| potrà essere:
\begin{lstlisting}[language=C++, style=codestyle]	
x.wait() {
	x_count++;
	
	if(next_count > 0) { // c'e' qualcuno in attesa
		signal(next);
	} else { // non c'e' nessuno, sblocca il monitor
		signal(mutex);
	}
	
	wait(x_sem);
	x_count--; // uscito da wait()
}
\end{lstlisting}

	\item Implementiamo quindi la \lstinline|x.signal()| secondo la politica signal and wait:
\begin{lstlisting}[language=C++, style=codestyle]	
x.signal() {
	if(x_count > 0) {
		signal(x_sem);

		next_count++;
		wait(next); // sono uno dei processi del monitor
		next_count--;
	} 
}
\end{lstlisting}

\end{itemize}

Ci dovrebbe quindi essere chiaro il funzionamento del monitor come un ambiente "ristretto" per i processi del sistema dove lo scheduling non è necessariamente FCFS (o qualsiasi fosse l'algoritmo usato dallo scheduler del sistema).

\subsubsection{Conditional wait}
Un'altra possibile politica che si può adottare all'interno dei monitor è la cosiddetta \textbf{conditional wait}, nella forma \lstinline|x.wait(c)| dove \lstinline|c| è un \textit{numero di priorità}. I processi con numero di priorità più piccolo (priorità più alta) vengono schedulati per primi.

Un esempio dove potrebbe essere utile usare tale costrutto è il seguente, dove si implementa un monitor con il compito di allocare una certa risorsa:
\begin{lstlisting}[language=C++, style=codestyle]	
monitor ResourceAllocator {
	boolean busy;
	condition x;
	void acquire(int time) {
		if(busy) x.wait(time);
		busy = TRUE;
	}

	void release() {
		busy = FALSE;
		x.signal();
	}
	
	initialization code() {
		busy = FALSE;
	}
}
\end{lstlisting}

In questo caso prendiamo come argomento \lstinline|time|, cioè il tempo per cui occupiamo la risorsa (meno tempo $\rightarrow$ più priorità).

\subsection{Deadlock}
Veniamo quindi alla trattazione vera e propria dei \textbf{deadlock}, o \textit{blocchi critici}, che avevamo introdotto in 13.2.

Di base, questi sono situazioni dove ciascun processo, in un insieme di processi, detiene una risorsa e ne desidera una di un altro.
Sul grafo di allocazione, equivalentemente, significa che abbiamo un \textit{ciclo}.

Ricordiamo che ci eravamo posti di implementare appropiate tecniche di deadlock \textbf{detection} (\textit{rilevamento} di deadlock) e deadlock \textbf{avoidance} (\textit{risoluzione} o \textit{prevenzione} di deadlock).

Una soluzione banale al problema del deadlock è obbligare il programmatore a dichiarare subito tutte le risorse di cui il programma ha bisogno: a questo punto, lato S/O, si potrà realizzare via mutex su tali risorse un sistema di bloccaggio che eviterà sempre i deadlock.
Il problema è chiaramente che tale vincolo è estremamente restrittivo, e renderebbe non solo molto scomodo per il programmatore programmare una data applicazione, ma in generale abbatterebbe l'efficienza dell'intero sistema.

\subsubsection{Condizioni di deadlock}
Esistono 4 condizioni necessarie affinché si verifichi un deadlock:
\begin{enumerate}
	\item \textbf{Mutua esclusione}: solo un processo per volta può usare una data risorsa;
	\item \textbf{Hold and wait}: un processo che ha ottenuto almeno una risorsa si mette in attesa di altre risorse ottenute da altri processi;
	\item \textbf{No preemption}: una risorsa può essere rilasciata solo \textit{volontariamente} dai processi che la ottengono, quando questi completano la loro operazione sulla stessa;
	\item \textbf{Attesa circolare}: esiste un insieme $\{p_0, p_1, ..., p_n \}$ di processi in attesa tali che $p_0$ aspetta una risorsa di $p_1$, $p_1$ aspetta una risorsa di $p_2$, ..., $p_{n - 1}$ aspetta una risorsa di $p_n$.
\end{enumerate}

\end{document}
